\newpage

\section{后记}

花了这么长时间终于写完了,期间经常熬到深夜,不得不感谢舍友的不杀之恩,通过这次大作业,对之前学的无约束优化算法有了更深的理解,感谢Stephen Boyd的\emph{Convex Optimization},给我提供了一种新的思路——以坐标变换的角度去审视最速下降法和牛顿法。

最觉得有成就感的还是第一题中关于最速下降法每次迭代的点都在两条直线上的分析,一开始是想找到在什么初始点能使线性收敛因子最大,后来凭借感觉觉得是在一条线上,再经过分析发现是在两条线上,然后发现对于一般的二次函数,迭代的点都固定在两条直线上,且这两条直线的方向向量矩阵关于$G^TG$共轭,最后还借此证明了线性收敛因子的上界。关于这些猜想,我先是上网搜然后翻书都没找到,最后无奈之下只好自己想,没想到真的发现了其中的规律,做完这一切之后真的感觉自己很棒很有成就感。

当然,这期间有喜悦也有痛苦,因为一个小问题导致debug到半夜甚至两三天的事简直不要太少,这里再次感谢舍友的不杀之恩,当然收获就是对于MATLAB的使用进步不小。

最后,啰啰嗦嗦了这么多只是觉得辛苦了这么久的成果不夹杂一点个人\sout{废话}情绪表达未免太对不起自己,就这样吧,结束了,我的大作业。

\vspace{3ex}

\begin{flushright}
张晋

\today

Mail: \href{15091060@buaa.edu.cn}{15091060@buaa.edu.cn}
\end{flushright}
%\rightline{}