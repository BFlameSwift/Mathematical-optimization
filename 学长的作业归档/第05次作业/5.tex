\documentclass[UTF8]{ctexart}
\usepackage{bm}
\usepackage{amssymb}
\usepackage{mathtools}
\usepackage{amsmath}
\usepackage{float}
\usepackage{rotating}
\usepackage{booktabs}
\title{\heiti 最优化第五次作业}
\author{\kaishu 张晋15091060}
\begin{document}
\maketitle
\begin{enumerate}
\item[2.24] 原始问题如下:

\begin{alignat}{2}
min \quad & x_1-x_2 \nonumber\\
\mbox{s.t.}\quad
&x_1\leq-1\nonumber\\
&x_2\geq 1\nonumber\\
&x_1\geq 0\nonumber\\
&x_2\geq 0
\end{alignat}

其对偶问题如下:

\begin{alignat}{2}
min \quad & -\lambda_1+\lambda_2 \nonumber\\
\mbox{s.t.}\quad
&\lambda_1\leq 0\nonumber\\
&\lambda_2\geq 0\nonumber\\
&\lambda_1\leq 1\nonumber\\
&\lambda_2\leq -1
\end{alignat}

显然,两个问题的可行解都不存在.

\clearpage
\item[2.25]
\textbf{充分性:}若存在$\bm{\lambda}\geq \bm{0}$,使得$\bm{c}^T+\bm{\lambda}^T\bm{A}= \bm{0}$成立,则有\[\bm{c}^T\bm{x}+\bm{\lambda}^T\bm{A}\bm{x}= \bm{0}\]
显然,如果$\bm{A}\bm{x}\leq \bm{0}$必然可以推出$\bm{c}^T\bm{x}\geq \bm{0}$.

\textbf{必要性:}对于线性规划:
\begin{alignat}{2}
min \quad & \bm{c}^T\bm{x} \nonumber\\
\mbox{s.t.}\quad &
\bm{A}\bm{x}\leq \bm{0} \nonumber\\
& \bm{x}\in \mathbb{R}^n
\end{alignat}

其对偶问题为:
\begin{alignat}{2}
max \quad & \bm{0} \nonumber\\
\mbox{s.t.}\quad &
\bm{\lambda}^T\bm{A}=\bm{c}^T \nonumber\\
&\bm{\lambda} \leq \bm{0}
\end{alignat}

对原始问题而言,显然$\bm{x}=\bm{0}$是一个可行解,且因为$\bm{c}^T\bm{x} \geq \bm{0}$,故原问题有下界,且下界在$\bm{x}=\bm{0}$处取到,根据弱对偶定理,该对偶问题也有解,则必存在$\bm{\lambda}\leq \bm{0}$,使得$\bm{\lambda}^T\bm{A}=\bm{c}^T$成立,那么将$\bm{\lambda}$取相反数,即可得$\bm{\lambda}\geq \bm{0}$满足$\bm{c}^T+\bm{\lambda}^T\bm{A}= \bm{0}$.


\textbf{几何解释:}
令$\bm{A}=[\bm{a}_1^T,\bm{a}_1^T,\cdots,\bm{a}_m^T]^T,\quad\bm{a}_i^T \in \mathbb{R}^n$

$\bm{A}\bm{x}\leq\bm{0}$是$n$维空间中由$m$个过原点的超平面$\bm{a}_1^T\bm{x}=\bm{0},\cdots,\bm{a}_m^T\bm{x}=\bm{0}$围成的凸多面锥体,而$\bm{c}^T\bm{x}\geq \bm{0}$是一个过原点的闭半空间,从直观角度来说,一个凸多面体锥很难包含一个闭半空间,除非这个凸多面体锥也展成了一张超平面,且与那个超平面重合.

在这种情况下,$rank(A)=1$,即整个向量$\bm{a}_1^T,\bm{a}_1^T,\cdots,\bm{a}_m^T$都能被$\bm{c}^T$线性表出,即存在$\mu_1,\mu_2,\cdots,\mu_m$,使得
$\mu_1\bm{a}_1^T=\mu_2\bm{a}_2^T=\cdots=\mu_m\bm{a}_m^T=\bm{c}^T$

\clearpage
\item[2.27]
\[P(\bm{x},\bm{y})=\bm{x}^T\bm{A}\bm{y}=\sum^{m}_{i=1}\sum^{n}_{j=1}x_iy_ja_{ij}\]
\begin{enumerate}
\item[(a)]
\[P(\bm{x},\bm{y})=\sum^{n}_{j=1}y_j\sum^{m}_{i=1}x_ia_{ij}\geq \alpha\sum^{n}_{j=1}y_j=\alpha\]

\item[(b)]
原约束转化如下:

\begin{alignat}{2}
max \quad & \alpha=[1,\bm{0}]\begin{bmatrix}
   	\alpha\\
    \bm{x}
\end{bmatrix} \nonumber\\
\mbox{s.t.}\quad &
[\alpha,\bm{x}^T]\begin{bmatrix}
   	0\\
    \bm{1}\end{bmatrix}=\bm{1}\nonumber\\
&[\alpha,\bm{x}^T]\begin{bmatrix}
   	1\\
    -\bm{A}\end{bmatrix}\leq  \bm{0}\nonumber\\
& x_i\geq 0
\end{alignat}

该问题的对偶问题为:
\begin{alignat}{2}
min \quad & [1,\bm{0}]\begin{bmatrix}
    	\beta\\
    \bm{y}
\end{bmatrix} \nonumber\\
\mbox{s.t.}\quad &
y_j\geq 0  \nonumber\\
&[0,\bm{1}]\begin{bmatrix}
   	\beta\\
    \bm{y}^T
\end{bmatrix}=\bm{1}\nonumber\\
& [\bm{1},-\bm{A}]\begin{bmatrix}
   	\beta\\
    \bm{y}^T\end{bmatrix}\geq \bm{0}
\end{alignat}
化简即可得证.

\item[(c)] 由于问题(5)(6)属于线性规划,由第一题知(5)有解且(6)有界,根据弱对偶性原理,可知(5)必有最优解,根据强对偶性原理,两个问题都有最优解且最优值相等.

\item[(d)] 若1表示正面,2表示反面,则:
\[\bm{A}=\begin{bmatrix}
   	1&-1\\
   	-1&1\end{bmatrix}\]

解题(a)的线性规划:
\begin{alignat}{2}
max \quad & \alpha\nonumber\\
\mbox{s.t.}\quad &
x_1+x_2=1\nonumber\\
&-x_1+x_2\geq \alpha\nonumber\\
&x_1-x_2\geq \alpha\nonumber\\
& x_1,x_2\geq 0
\end{alignat}
求得最优解为$\bm{x}=(0.5,0.5)$,最优值为0,可见该点即是平衡点.

\item[(e)]
\[\bm{A}=\begin{bmatrix}
   	0&3&-1\\
-3&0&3\\
1&-3&0\end{bmatrix}\]

解题(a)的线性规划:
\begin{alignat}{2}
max \quad & \alpha\nonumber\\
\mbox{s.t.}\quad &
x_1+x_2+x_3=1\nonumber\\
&-3x_2+x_3\geq \alpha\nonumber\\
&3x_1-3x_3\geq \alpha\nonumber\\
&-x_1+3x_3\geq \alpha\nonumber\\
& x_1,x_2,x_3\geq 0
\end{alignat}
求得最优解为$\bm{x}=(3/7,1/7,3/7)$,最优值为0,可见该点即是平衡点.
\end{enumerate}
\end{enumerate}
\end{document}

