\documentclass[UTF8]{ctexart}
\usepackage{bm}
\usepackage{amssymb}
\usepackage{mathtools}
\usepackage{amsmath}
\usepackage{float}
\usepackage{rotating}
\usepackage{booktabs}
\usepackage{pdfpages}

\title{\heiti 最优化第九次作业}
\author{\kaishu 张晋15091060}
\begin{document}
\maketitle
\begin{enumerate}
\item[1.4]
\begin{enumerate}
\item $\nabla (\bm{a}^T\bm{x})=(\nabla \bm{a}^T)\bm{x}+(\nabla \bm{x}^T)\bm{a}=\bm{a}$

$\nabla ^2(\bm{a}^T\bm{x})=\nabla \bm{a}=0$

\item $\nabla (\bm{x}^T\bm{A}\bm{x})=(\nabla \bm{x}^T\bm{A})\bm{x}+(\nabla \bm{x}^T)(\bm{x}^T\bm{A})^T=(\bm{A}+\bm{A}^T)\bm{x}$

$\nabla ^2(\bm{x}^T\bm{A}\bm{x})=\bm{A}+\bm{A}^T$

\item $\nabla (\dfrac{1}{2}\bm{x}^T\bm{A}\bm{x}-\bm{b}^T\bm{x})=\dfrac{1}{2}(\bm{A}+\bm{A}^T)\bm{x}-\bm{b}=\bm{A}\bm{x}-\bm{b}$

 $\nabla^2(\dfrac{1}{2}\bm{x}^T\bm{A}\bm{x}-\bm{b}^T\bm{x})=\nabla(\bm{A}\bm{x}-\bm{b})=\bm{A}$
 
\item $\nabla (\bm{r}^T\bm{r})=(\nabla \bm{r}^T)\bm{r}+(\nabla \bm{r}^T)\bm{r}=2\bm{A}^T\bm{r}$

$\nabla^2 (\bm{r}^T\bm{r})=\nabla(2\bm{A}^T\bm{r})=2(\nabla \bm{A}^T)\bm{r}+2\bm{A}^T\bm{A}$
\end{enumerate}

\item[1.6]
\begin{equation}
\bm{f}(\bm{x})=\bm{f}(\bm{x'})+
(\bm{x}-\bm{x}')^T\nabla \bm{f}(\bm{x}')+\dfrac{1}{2}(\bm{x}-\bm{x}')^T\bm{A}(\bm{x}-\bm{x}')+o(\left \| \bm{x}-\bm{x}'\right\|^2)
\end{equation}

\item[1.7]
函数$\bm{g}$沿$\bm{p}$方向的斜率为$\bm{p}^T\nabla \bm{g}(x')=\bm{p}^T\bm{g}'$
\[\bm{p}^T\bm{g}'=\big \|\bm{p}^T\bm{g}'\big\|_2\leq \big\|\bm{p}^T\big\|_2\cdot \big\|\bm{g}'\big\|_2=\big\|\bm{g}'\big\|_2\]

故当$\bm{p}(\bm{g}')^T=\bm{p}^T\bm{g}'= \big\|\bm{g}'\big\|_2$时,斜率取到最大值$\big\|\bm{g}'\big\|_2$,此时$\bm{p}(\bm{g}')^T\bm{g}'=\bm{g}'\big\|\bm{g}'\big\|_2$,得$\bm{p}=\bm{g}'/\big\|\bm{g}'\big\|_2$

\newpage
\item[4.2]
\begin{enumerate}
\item $\mathbb{R}^n$空间经过$\bm{A}$变换后形成的新的m维空间到点$\bm{B}$的最短欧式距离。

\item 设$g(\bm{x})=\|\bm{Ax}-\bm{b}\|^2$
\begin{align}
\nabla g&=\nabla \|\bm{Ax}-\bm{b}\|^2\\
&=\nabla (\bm{Ax}-\bm{b})^T(\bm{Ax}-\bm{b})\\
&=2[\nabla(\bm{x}^T\bm{A}^T-\bm{b}^T)](\bm{Ax}-\bm{b})\\
&=2\bm{A}^T(\bm{Ax}-\bm{b}) 
\end{align}

$\nabla^2 g=2\bm{A}^T\bm{A}$为半正定矩阵,故其必要条件是$\bm{A}^T(\bm{Ax}-b)=\bm{0}$.显然,它也是充分条件。

\item 不唯一,满足$\bm{A}^T\bm{Ax}=\bm{A}^T\bm{b}$的解$\bm{x}$都是最优解,当$rank(\bm{A}^T\bm{A})\leq n$时可能会出现多解.

\item 若$\bm{A}^T\bm{A}$是严格正定的,则最优解$\bm{x}=(\bm{A}^T\bm{A})^{-1}\bm{A}^T\bm{b}$

\item 
\[\bm{A}^T\bm{A}=\begin{bmatrix}
 2 & -1 & 1 \\
 -1 & 6 & 2 \\
 1 & 2 & 2 
\end{bmatrix},\qquad 
\bm{x}=(\bm{A}^T\bm{A})^{-1}\bm{A}^T\bm{b}=\begin{bmatrix}
3 \\
3/2 \\
-5/2
\end{bmatrix}\]
\end{enumerate}

\newpage
\item[4.3]
\[\nabla q(\bm{x})=\dfrac{1}{2}(\bm{G}+\bm{G}^T)\bm{x}-\bm{b}\]
\[\nabla^2 q(\bm{x})=\dfrac{1}{2}(\bm{G}+\bm{G}^T) \]
此处题目中没写$\bm{G}$是否为对称阵,如果$\bm{G}$不是对称阵,那么第一问的结论将不成立(下面给出证明),感觉题目的意思应该是约定俗成了$\bm{G}$是对称阵,那么此题我继续按照已知$\bm{G}$是对称阵的条件去做。
\begin{enumerate}
\item (若$\bm{G}$为对称阵)$\nabla q(\bm{x})=\bm{G}\bm{x}-\bm{b},\nabla^2 q(\bm{x})=\bm{G}$,若$\bm{G}$半正定,$\bm{G}\bm{x}=\bm{b}$有解,那么该解$\bm{x}^{\ast}$即为极小点。同时若$\bm{x}^{\ast}$为极小点,那么也必同时包含以上条件。

(在没有$\bm{G}$为对称阵的条件下):若$\bm{G}$半正定,且$\bm{G}\bm{x}=\bm{b}$有解,则容易得到$\bm{G}+\bm{G}^T$半正定的结论,但无法推出$\dfrac{1}{2}(\bm{G}+\bm{G}^T)\bm{x}=\bm{b}$有解的结论。可以举出例子如下:
\[\bm{G}=\begin{bmatrix}
   	1&0\\
   	2&1\end{bmatrix}\text{满足半正定条件},\quad 
\bm{b}=\begin{bmatrix}
   	1\\
   2\end{bmatrix},\text{解得}:
\bm{x}=\bm{G}^{-1}\bm{b}=\begin{bmatrix}
   	1\\
   2\end{bmatrix}\]
\[\text{但对于方程}\dfrac{1}{2}(\bm{G}+\bm{G}^T)\bm{x}=\bm{b}:\quad 
\begin{bmatrix}
   	1&1\\
   	1&1\end{bmatrix}\bm{x}=\begin{bmatrix}
   	1\\
   2\end{bmatrix}\text{是无解的.}\]

同时,$\dfrac{1}{2}(\bm{G}+\bm{G}^T)\bm{x}=\bm{b}$有解、$\dfrac{1}{2}(\bm{G}+\bm{G}^T)$半正定的条件也无法推出$\bm{G}\bm{x}=b.$举例如下:
\[\bm{G}=\begin{bmatrix}
   	2&0\\
   	1&0\end{bmatrix},\quad 
\dfrac{1}{2}(\bm{G}+\bm{G}^T)=\begin{bmatrix}
   	2&1/2\\
   	1/2&0\end{bmatrix}\text{满足半正定条件},\]
\[\text{若}b=\begin{bmatrix}
   	1\\
   2\end{bmatrix},\text{则}
\dfrac{1}{2}(\bm{G}+\bm{G}^T)\bm{x}=b\text{有解,但}
\bm{G}\bm{x}=b:\quad \begin{bmatrix}
   	2&0\\
   	1&0\end{bmatrix}\bm{x}=
\begin{bmatrix}
   	1\\
   2\end{bmatrix}\text{无解.}\]

\item $\bm{G}$正定时,$q(\bm{x})$严格为凸函数,方程$\bm{G}\bm{x}=\bm{b}$有唯一解$\bm{x}=\bm{G}^{-1}\bm{b}$,该点即为唯一的极小点.

\item 若$\bm{G}$半正定,$q(\bm{x})$为凸函数,满足方程$\bm{G}\bm{x}=\bm{b}$的点都为局部极小点,由凸性可知每个局部极小点都为全局极小点.


\end{enumerate}

\newpage
\item[4.6]
先证必要性:若$f$为凸函数,则对$\forall x_1,x_n \in \mathbb{R}^n$,有
\begin{equation}
f(\theta_1x_1+(1-\theta_1)x_n)\leq \theta_1f(x_1)+(1-\theta_1)f(x_n)
\end{equation}

那么对于
\begin{equation}
f(\theta_1x_1+\theta_2x_2+(1-\theta_1-\theta_2)x_n)
\end{equation}
令
\begin{equation}
\qquad x'=\dfrac{\theta_1x_1+\theta_2x_2}{\theta_1+\theta_2}
\end{equation}

代入式(7)得:

\begin{equation}
f((\theta_1+\theta_2)x'+(1-\theta_1-\theta_2)x_n)\leq (\theta_1+\theta_2)f(x')+(1-\theta_1-\theta_2)f(x_n)
\end{equation}

根据凸函数性质(6),有:
\begin{equation}
f(x')=(\dfrac{\theta_1}{\theta_1+\theta_2}x_1+\dfrac{\theta_2}{\theta_1+\theta_2}x_2)\leq \dfrac{\theta_1}{\theta_1+\theta_2}f(x_1)+\dfrac{\theta_2}{\theta_1+\theta_2}f(x_2)
\end{equation}

将式(10)代入式(9)得:
\begin{equation}
f(\theta_1x_1+\theta_2x_2+(1-\theta_1-\theta_2)x_n)\leq 
\theta_1f(x_1)+\theta_2f(x_2)+(1-\theta_1-\theta_2)f(x_n)
\end{equation}

同理,可将该不等式从2扩充到任意k,即为所证。

取$k=2$,充分性即可得证.






\end{enumerate}
\end{document}