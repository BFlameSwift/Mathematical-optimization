\documentclass[UTF8]{ctexart}
\usepackage[table]{xcolor}
\usepackage{bm}
\usepackage{amssymb}
\usepackage{mathtools}
\usepackage{amsmath}
\usepackage{float}
\usepackage{rotating}
\usepackage{booktabs}
\usepackage{pdfpages}
\usepackage{subfigure}
\usepackage{mathtools}
\usepackage{amsmath}
\usepackage{listings}
\usepackage{fancybox}
%\usepackage{xcolor}
%\usepackage{colortbl}
\usepackage{diagbox}
\usepackage{amssymb}
\usepackage{warpcol}
\usepackage{lscape}
\usepackage[framemethod=tikz]{mdframed}
\usepackage{longtable,booktabs}
\definecolor{ocre}{RGB}{243,102,25}
\definecolor{mygray}{RGB}{243,243,244}


\lstset{
columns=flexible,
numbers=left,
numberstyle=\footnotesize\color{darkgray}, 
basicstyle=\small\ttfamily,
stringstyle=\color{purple},
keywordstyle=\color[RGB]{40,40,255}\bfseries,
commentstyle=\it\color[RGB]{0,96,96},  
stringstyle=\rmfamily\slshape\color[RGB]{128,0,0}, 
showstringspaces=false,      
% directivestyle=\color{blue},
frame=shadowbox,
%framerule=0pt,
backgroundcolor=\color[RGB]{245,245,244},
escapeinside=``, %逃逸字符(1左面的键),用于显示中文
breaklines,
extendedchars=false,
%解决代码跨页时,章节标题,页眉等汉字不显示的问题
xleftmargin=2em,xrightmargin=2em,
aboveskip=1em,%设置边距
tabsize=4, %设置tab空格数  
showspaces=false %不显示空格 
rulesepcolor=\color{red!20!green!20!blue!20}
%rulesepcolor=\color{brown}
}

\title{\heiti 最优化第十三次作业}
\author{\kaishu 张晋15091060}
\begin{document}
\maketitle
\begin{enumerate}
\item[5.22]
\begin{enumerate}
\item
由$f$连续可微且严格凸可知:
\[g[x^{(k+1)}]> g[x^{(k)}]+\nabla^2f(x^{(k)})[x^{(k+1)}-x^{(k)}]\]
于是有:
\begin{align}
{s^{(k)}}^Ty^{(k)}&=[x^{(k+1)}-x^{(k)}][g^{(k+1)}-g^{(k)}]\\
&>[x^{(k+1)}-x^{(k)}]^T\nabla^2f(x^{(k)})[x^{(k+1)}-x^{(k)}]>0
\end{align}
\item 
设$f(x)=-x^2+\dfrac{7}{4}x-1$,其满足$f(0)=-1,f(1)=-1/4$,且$g(x)=-2x+\dfrac{7}{4},\quad s^{(k)}=1-0=1,\quad y^{(k)}=g(1)-g(0)=-2,\quad {s^{(k)}}^Ty^{(k)}=-2<0.$
\end{enumerate}

\item[5.23] Matlab程序结果运行如下,验证了题干命题:

\begin{lstlisting}[language=Octave]
step =
     2

x =
     0
     0

ans =
     0

H =
    0.0500   -0.0000
   -0.0000    0.5000
\end{lstlisting}

\newpage
代码如下:
\begin{lstlisting}[language=Octave]
%   5.23:   精确步长的DFP法
clc;
clear;
N=200;
step=0;
x=[0.1,1]';
H=eye(2);
e=0.000001;
g=gg(x);
G=[20,0;0,2];
while (norm(g)>e  && step < N) 
	step=step+1;
	p=-H*g;
	s=Alpha(p,g,G)*p;
	y=gg(x+s)-g;
	x=x+s;
	g=gg(x);
	H=H_update(H,s,y);
end
step
x
f(x)
H

function y=f(x)
y=10*x(1)^2+x(2)^2;
end
function y=gg(x)
y=[20*x(1),2*x(2)]';
end
function h=H_update(H,s,y)
h=H+(s*s')/(s'*y)-(H*y*y'*H)/(y'*H*y);
end 
function a=Alpha(p,g,G)
a=-(p'*g)/(p'*G*p);
a=double(a);
end
\end{lstlisting}
\end{enumerate}
\end{document}